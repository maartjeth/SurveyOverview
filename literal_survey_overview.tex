% !TEX root = ../main.tex
\onecolumn

\section{Verbatim survey overview}
\label{sec:appendix_literal_survey_overview}

\addtocounter{footnote}{1}
\footnotetext{\url{http://surveymonkey.com}}
\addtocounter{footnote}{-1}

\tablehead{
\toprule
\textbf{Question Nr.} & \textbf{Question and Answer Options} \\
\midrule
}
\captionof{table}{A complete overview of the survey. This table includes the explanation that participants received, as well as all the questions and the answer options. If a question was the start of a branch, the direction of the branch has been written behind the answer options in italic. (This was never shown to the participants.) Note that the survey was performed in SurveyMonkey.\protect\footnote{\protect\url{http://surveymonkey.com}} The survey had a lay-out as provided by SurveyMonkey, i.e., it consisted of different pages and colors were used to highlight certain important parts in texts.}
\begin{xtabular}{@{\,}l <{\hskip 2pt} >{\raggedright\arraybackslash}p{0.82\textwidth}}

    Q1  & \textbf{Introduction and Instructions} \\

        & Thank you for taking the time to fill out this survey! Before you start, please take the time to read these instructions carefully. If you still have any questions after reading the instructions, please send them to [anonymized]. \\
        & We will give away 10 [anonymized] vouchers of 10 [anonymized currency] each among the participants. If you would like to take part in the raffle, you can leave your email address at the end of this survey. \\

        & \textbf{Goal of the study} \\

        & The goal of this survey is to get insight in how summaries help or can help you when studying. \\

        & \textbf{What the survey will look like} \\

        & In what follows you will get questions that aim to develop an understanding for:
        \begin{itemize}[leftmargin=*]
          \item For which types of study material it is useful to have summaries
          \item How these summaries can help you with your task
          \item What these summaries should look like
        \end{itemize} \\

        & We expect this survey to take approximately 10 minutes of your time. \\

        &  Use the next button to go to the next page once you have filled out all the questions on the page. Use the prev button to go back one page. \\

        & \textbf{About your privacy} \\

        & We value your privacy and will process your answers anonymously. The answers of all participants in this survey will be used to gain insight in how pre-made summaries can be helpful for different types of studying activities. The answers will be presented in a research paper about this topic. This will be done either in an aggregated manner, or by citing verbatim examples of the answers. Again, this will all be done anonymously. \\

        & \\

        & \textbf{I agree that I have read and understood the instructions. I also understand that my participation in this survey is voluntarily.}
        \begin{itemize}[label=$\square$, leftmargin=*]
            \item I agree
        \end{itemize} \\

    Q2  & \textbf{Important! Some background knowledge you need to know} \\

        & Throughout this survey we make use of the term pre-made summary. It is very important that you understand what this means. On this page we explain this term, so please make sure to read this carefully. \\

        & \textbf{Definition pre-made summary} \\

        & One type of summary is one that you make yourself. Another type of summary is one that has been made for you. In this survey, we focus on this latter type and we call them pre-made summaries. \\

        & \textbf{Who makes these pre-made summaries?} \\

        & These pre-made summaries can be made by a person, for example your teacher, your friend, a fellow student or someone at some official organisation, etc. The pre-made summaries can also be made by a computer. \\

        & \textbf{What kinds of summaries are we talking about?} \\

        & There are no restrictions on what these pre-made summaries can look like. On the contrary, that is one of the things we aim to find out with this survey! But, to give some examples, you could think of a written overview of a text book, highlights in text to draw your attention to important bits, blog posts, etc. These are really just examples and don't let them limit your creativity! You can come up with any example of a pre-made summary that is helpful for you. \\

        & \\

        & \textbf{Yes, I understand what a pre-made summary is!}
          \begin{itemize}[label=$\square$, leftmargin=*, nosep]
            \item Yes
          \end{itemize} \\

        Q3 & Please think back to your recent study activities. Examples of study activities can be: studying for an exam, writing a paper, doing homework exercises, etc. Note that these are just examples, any other study activity is fine too. \\

        & \textbf{Did you use a pre-made summary in any of these study activities?}
        \begin{itemize}[label=$\square$, leftmargin=*, nosep]
          \item Yes \textit{-- participants are led to Q6}
          \item No \textit{-- participants are led to Q4}
        \end{itemize} \\

        Q4 & \textbf{Can you think of one of your recent study activities where a pre-made summary would have been useful for you?}
        \begin{itemize}[label=$\square$, leftmargin=*, nosep]
          \item Yes \textit{-- participants are led to Q25}
          \item No \textit{-- participants are led to Q5}
        \end{itemize}

        \\

        Q5 & \textbf{Why do you think a pre-made summary would not have helped you with any of your recent study activities? }\\

        & Open response \textit{-- participants are led to Q48}

        \\ \midrule

        \multicolumn{2}{l}{\textbf{Start branch of participants who described an existing summary}} \\ \midrule

        & If you have multiple study activities where you used a pre-made summary, please take the one where you found the pre-made summary most useful. \\

        Q6 & \textbf{The original study material consisted of}
        \begin{itemize}[label=$\square$, leftmargin=*, nosep]
          \item Mainly text \textit{-- participants are led to Q8}
          \item Mainly figures \textit{-- participants are led to Q7}
          \item Mainly video \textit{-- participants are led to Q7}
          \item Mainly audio \textit{-- participants are led to Q7}
          \item A combination of some or all of the above \textit{-- participants are led to Q7}
          \item I do not know, because I have not seen the study material \textit{-- participants are led to Q7}
          \item Other (please specify) \textit{-- participants are led to Q7}
        \end{itemize}

        \\

        Q7 & \textbf{For now we narrow down our survey to study material that is mostly textual. Do you recall any other recent study activity where you made use of a pre-made summary and where the original study material mainly consisted of text?}
        \begin{itemize}[label=$\square$, leftmargin=*, nosep]
          \item Yes \textit{-- participants are led to Q8}
          \item No \textit{-- participants are led to Q48}
        \end{itemize}

        \\

        Q8 & \textbf{What was the goal of this study activity?}
        \begin{itemize}[label=$\square$, leftmargin=*, nosep]
          \item Studying for an exam
          \item Writing a paper / essay / report / etc.
          \item Doing homework exercises
          \item Other (please specify)
        \end{itemize}

        \\

        Q9 & \textbf{Who made this pre-made summary?}
        \begin{itemize}[label=$\square$, leftmargin=*, nosep]
          \item A teacher or teaching assistant
          \item A fellow student
          \item An official organisation
          \item The authors of the original study material
          \item A computer program
          \item I am not sure, I found it online
          \item Other (please specify)
        \end{itemize}

        \\

        & Now some questions will follow about what the study material that was summarized looked like. \\

        Q10 & \textbf{What was the length of the study material?}
        \begin{itemize}[label=$\square$, leftmargin=*, nosep]
          \item A single article
          \item Multiple articles
          \item A single book chapter
          \item Multiple book chapters from the same book
          \item Multiple book chapters from various books
          \item A combination of the above
          \item I do not know because I have not seen the study material, only the summary
          \item Other (please specify)
        \end{itemize}

        \\

        Q11 & \textbf{How was the study material structured? (Multiple answers possible)}
        \begin{itemize}[label=$\square$, leftmargin=*, nosep]
          \item There was no particular structure - e.g. just one large text
          \item The text contained a title or titles
          \item The text contained subheadings
          \item The text consisted of different chapters
          \item The text consisted of different sections and / or paragraphs
          \item I do not know because I have not seen the study material, only the summary
          \item Other (please specify)
        \end{itemize}

        \\

        Q12 & \textbf{What was the genre of the study material?}
        \begin{itemize}[label=$\square$, leftmargin=*, nosep]
          \item Mainly educational (such as a text book (chapter))
          \item Mainly scientific (such as an academic article, publication, report, etc)
          \item Mainly nonfiction writing (such as (auto)biographies, history books, etc)
          \item Mainly fiction writing (such as novels, short fictional stories, etc)
          \item Other (please specify)
        \end{itemize}

        \\

        Q13 & \textbf{How would you classify the difficulty level of the study material?}
        \begin{itemize}[label=$\square$, leftmargin=*, nosep]
          \item Ordinary: most people would be able to understand it
          \item Specialized: you need to know the jargon of the field to be able to understand it
          \item Geographically based: you can only understand it if you are familiar with a certain area, for example because it contains local names
        \end{itemize}

        \\

        & Now we will ask some questions about the purpose of the pre-made summary that you used. \\

        Q14 & \textbf{The summary was made specifically to help me (and potentially fellow students) with my study activity.}

        \begin{tabularx}{0.7\columnwidth}{
          !{\hskip 2pt}>{\centering\arraybackslash}p{0.11\columnwidth}
          !{\hskip 2pt}>{\centering\arraybackslash}p{0.11\columnwidth}
          !{\hskip 2pt}>{\centering\arraybackslash}p{0.11\columnwidth}
          !{\hskip 2pt}>{\centering\arraybackslash}p{0.11\columnwidth}
          !{\hskip 2pt}>{\centering\arraybackslash}p{0.11\columnwidth}
          !{\hskip 2pt}>{\centering\arraybackslash}p{0.11\columnwidth}}

          Strongly disagree & Disagree & Neither agree nor disagree & Agree & Strongly agree & I don't know \\

          $\square$ & $\square$ & $\square$ & $\square$ & $\square$ & $\square$
        \end{tabularx}

        \\

        Q15 & \textbf{For what type of people was the summary intended? Your score can range from (1) Untargetted, to (5) Targetted.}

        \begin{tabularx}{0.7\columnwidth}{
          !{\hskip 2pt}>{\centering\arraybackslash}p{0.13\columnwidth}
          !{\hskip 2pt}>{\centering\arraybackslash}p{0.13\columnwidth}
          !{\hskip 2pt}>{\centering\arraybackslash}p{0.13\columnwidth}
          !{\hskip 2pt}>{\centering\arraybackslash}p{0.13\columnwidth}
          !{\hskip 2pt}>{\centering\arraybackslash}p{0.13\columnwidth}}

          Untargetted: No domain knowledge is expected from the users of the summmary. & & & & Targetted: Full domain knowledge is expected from the users of the summmary. \\
          (1) & (2) & (3) & (4) & (5) \\

          $\square$ & $\square$ & $\square$ & $\square$ & $\square$
        \end{tabularx}

        \\

        Q16 & \textbf{How did this summary help you with your task? (Multiple answers possible)}
        \begin{itemize}[label=$\square$, leftmargin=*, nosep]
          \item The summary helped to retrieve parts of the original study material
          \item I used the summary to preview the text that I was about to read
          \item I used the summary as a substitute for the original study material
          \item I used the summary to refresh my memory of the original study material
          \item I used the summary as a reminder that I had to read the original study material
          \item The summary helped to get an overview of the original study material
          \item The summary helped to understand the original study material
          \item Other (please specify)
        \end{itemize}

        \\

        Q17 & \textbf{What was the type of the summary?}
        \begin{itemize}[label=$\square$, leftmargin=*, nosep]
          \item Lecture notes
          \item Blog post
          \item Highlights of some kind in the original study material
          \item Abstractive piece of text, such as a written overview of a text book, an abstract of a paper, etc.
          \item Short video
          \item A slide show
          \item Other (please specify)
        \end{itemize}

        \\

        Q18 & \textbf{How was the summary structured? (Multiple answers possible)}
        \begin{itemize}[label=$\square$, leftmargin=*, nosep]
          \item The summary was a running text, without particular structure
          \item The summary consisted of highlights in the original study material, without particular structure
          \item The summary itself contained special formatting, such as bold or cursive text, highlights, etc
          \item The summary contained diagrams
          \item The summary contained tables
          \item The summary contained graphs
          \item The summary contained figures
          \item The summary contained headings
          \item The summary consisted of different sections / paragraphs
          \item Other (please specify)
        \end{itemize}

        \\

        Q19 & \textbf{How much of the study material was covered by the summary?}
        \begin{tabularx}{0.7\columnwidth}{
          !{\hskip 2pt}>{\centering\arraybackslash}p{0.13\columnwidth}
          !{\hskip 2pt}>{\centering\arraybackslash}p{0.13\columnwidth}
          !{\hskip 2pt}>{\centering\arraybackslash}p{0.13\columnwidth}
          !{\hskip 2pt}>{\centering\arraybackslash}p{0.13\columnwidth}
          !{\hskip 2pt}>{\centering\arraybackslash}p{0.13\columnwidth}}

          None of the study material was covered &
          Almost none of the study material was covered &
          Some of the study material was covered &
          Most of the study material was covered &
          All of the study material was covered \\
          (1) & (2) & (3) & (4) & (5) \\

          $\square$ & $\square$ & $\square$ & $\square$ & $\square$
        \end{tabularx}

        \\

        Q20 & \textbf{What was the style of this summary?}
        \begin{itemize}[label=$\square$, leftmargin=*, nosep]
          \item Informative: the summary simply conveyed the information that was in the original study material
          \item Indicative: the summary gave an idea of the topic of the study material, but not more
          \item Critical: the summary gave a critical review of the study material
          \item Aggregative: the summary put different source texts in relation to one another and by doing this gave an overview of a certain topic
          \item Other (please specify)
        \end{itemize}

        \\

        Q21 & \textbf{Overall, how helpful was the pre-made summary for you? Your score can range from (1) Not helpful at all, to (5) Very helpful.}

        \begin{tabularx}{0.7\columnwidth}{
          !{\hskip 2pt}>{\centering\arraybackslash}p{0.13\columnwidth}
          !{\hskip 2pt}>{\centering\arraybackslash}p{0.13\columnwidth}
          !{\hskip 2pt}>{\centering\arraybackslash}p{0.13\columnwidth}
          !{\hskip 2pt}>{\centering\arraybackslash}p{0.13\columnwidth}
          !{\hskip 2pt}>{\centering\arraybackslash}p{0.13\columnwidth}}

          Not helpful at all & & & & Very helpful \\
          (1) & (2) & (3) & (4) & (5) \\

          $\square$ & $\square$ & $\square$ & $\square$ & $\square$
        \end{tabularx}

        \\

        Q22 & \textbf{Imagine you could turn this summary into your ideal summary. What would you change?} \\

        & Open response

        \\

        Q23 & \textbf{Is there anything else you want us to know about the summary that we have not covered yet?} \\

        & Open response

        \\

        Q24 & \textbf{Is there anything else you want us to know about the original study material that we have not covered yet?} \\

        & Open response \textit{-- participants are led to Q40}

        \\ \midrule

        \multicolumn{2}{l}{\textbf{Start branch of participants who described an imagined summary}} \\ \midrule

        & Please take one of these study activities in mind and imagine you would have had a pre-made summary. \\

        Q25 & \textbf{The original study material consisted of}
        \begin{itemize}[label=$\square$, leftmargin=*, nosep]
          \item Mainly text \textit{-- participants are led to Q27}
          \item Mainly figures \textit{-- participants are led to Q26}
          \item Mainly video \textit{-- participants are led to Q26}
          \item Mainly audio \textit{-- participants are led to Q26}
          \item A combination of some or all of the above \textit{-- participants are led to Q26}
          \item Other (please specify) \textit{-- participants are led to Q26}
        \end{itemize}

        \\

        Q26 & \textbf{For now we narrow down our survey to study material that is mostly textual. Do you recall any other recent study activity where you could have used a pre-made summary and where the original study material mainly consisted of text?}
        \begin{itemize}[label=$\square$, leftmargin=*, nosep]
          \item Yes \textit{-- participants are led to Q27}
          \item No \textit{-- participants are led to Q48}
        \end{itemize}

        \\

        Q27 & \textbf{What was the goal of this study activity?}
        \begin{itemize}[label=$\square$, leftmargin=*, nosep]
          \item Studying for an exam
          \item Writing a paper / essay / report / etc.
          \item Doing homework exercises
          \item Other (please specify)
        \end{itemize}

        \\

        & Now some questions will follow about what the study material that could be summarized looked like. \\

        Q28 & \textbf{What was the length of the study material?}
        \begin{itemize}[label=$\square$, leftmargin=*, nosep]
          \item A single article
          \item Multiple articles
          \item A single book chapter
          \item Multiple book chapters from the same book
          \item Multiple book chapters from various books
          \item A combination of the above
          \item Other (please specify)
        \end{itemize}

        \\

        Q29 & \textbf{How was the study material structured? (Multiple answers possible)}
        \begin{itemize}[label=$\square$, leftmargin=*, nosep]
          \item There was no particular structure - e.g. just one large text
          \item The text contained a title or titles
          \item The text contained subheadings
          \item The text consisted of different chapters
          \item The text consisted of different sections and / or paragraphs
          \item Other (please specify)
        \end{itemize}

        \\

        Q30 & \textbf{What was the genre of the study material?}
        \begin{itemize}[label=$\square$, leftmargin=*, nosep]
          \item Mainly educational (such as a text book (chapter))
          \item Mainly scientific (such as an academic article, publication, report, etc)
          \item Mainly nonfiction writing (such as (auto)biographies, history books, etc)
          \item Mainly fiction writing (such as novels, short fictional stories, etc)
          \item Other (please specify)
        \end{itemize}

        \\

        Q31 & \textbf{How would you classify the difficulty level of the study material?}
        \begin{itemize}[label=$\square$, leftmargin=*, nosep]
          \item Ordinary: most people would be able to understand it
          \item Specialized: you need to know the jargon of the field to be able to understand it
          \item Geographically based: you can only understand it if you are familiar with a certain area, for example because it contains local names
        \end{itemize}

        \\

        & Now we will ask some questions about the purpose of the pre-made summary that would have been helpful. \\

        Q32 & \textbf{For what type of people should the summary ideally be intended? Your score can range from (1) Untargetted, to (5) Targetted.}
        \begin{tabularx}{0.7\columnwidth}{
          !{\hskip 2pt}>{\centering\arraybackslash}p{0.13\columnwidth}
          !{\hskip 2pt}>{\centering\arraybackslash}p{0.13\columnwidth}
          !{\hskip 2pt}>{\centering\arraybackslash}p{0.13\columnwidth}
          !{\hskip 2pt}>{\centering\arraybackslash}p{0.13\columnwidth}
          !{\hskip 2pt}>{\centering\arraybackslash}p{0.13\columnwidth}}

          Untargetted: No domain knowledge is expected from the users of the summmary. & & & & Targetted: Full domain knowledge is expected from the users of the summmary. \\
          (1) & (2) & (3) & (4) & (5) \\

          $\square$ & $\square$ & $\square$ & $\square$ & $\square$
        \end{tabularx}

        \\

        Q33 & \textbf{How would this summary help you with your task? (Multiple answers possible)}
        \begin{itemize}[label=$\square$, leftmargin=*, nosep]
          \item The summary would help to retrieve parts of the original study material
          \item I would use the summary to preview the text that I was about to read
          \item I would use the summary as a substitute for the original study material
          \item I would use the summary to refresh my memory of the original study material
          \item I would use the summary as a reminder that I had to read the original study material
          \item The summary would help to get an overview of the original study material
          \item The summary would help to understand the original study material',
          \item Other (please specify)
        \end{itemize}

        \\

        & Now we will ask some questions about what the summary should look like and cover. \\

        Q34 & \textbf{What would be the ideal type of the summary?}
        \begin{itemize}[label=$\square$, leftmargin=*, nosep]
          \item Lecture notes
          \item Blog post
          \item Highlights of some kind in the original study material
          \item Abstractive piece of text, such as a written overview of a text book, an abstract of a paper, etc.
          \item Short video
          \item A slide show
          \item Other (please specify)
        \end{itemize}

        \\

        Q35 & \textbf{What is the ideal structure of the summary? (Multiple answers possible)}
        \begin{itemize}[label=$\square$, leftmargin=*, nosep]
          \item The summary should be a running text, without particular structure
          \item The summary should consist of highlights in the original study material, without particular structure
          \item The summary itself should contain special formatting, such as bold or cursive text, highlights, etc.
          \item The summary should contain diagrams
          \item The summary should contain tables
          \item The summary should contain graphs
          \item The summary should contain figures
          \item The summary should contain headings
          \item The summary should consist of different sections / paragraphs
          \item Other (please specify)
        \end{itemize}

        \\

        Q36 & \textbf{How much of the study material should be covered by the summary?}
        \begin{tabularx}{0.7\columnwidth}{
          !{\hskip 2pt}>{\centering\arraybackslash}p{0.13\columnwidth}
          !{\hskip 2pt}>{\centering\arraybackslash}p{0.13\columnwidth}
          !{\hskip 2pt}>{\centering\arraybackslash}p{0.13\columnwidth}
          !{\hskip 2pt}>{\centering\arraybackslash}p{0.13\columnwidth}
          !{\hskip 2pt}>{\centering\arraybackslash}p{0.13\columnwidth}}

          None of the study material should be covered &
          Almost none of the study material should be covered &
          Some of the study material should be covered &
          Most of the study material should be covered &
          All of the study material should be covered \\
          (1) & (2) & (3) & (4) & (5) \\
          $\square$ & $\square$ & $\square$ & $\square$ & $\square$
        \end{tabularx}

        \\

        Q37 & \textbf{What should the style of this summary be?}
        \begin{itemize}[label=$\square$, leftmargin=*, nosep]
          \item Informative: the summary should simply convey the information that was in the original study material
          \item Indicative: the summary should give an idea of the topic of the study material, but not more
          \item Critical: the summary should give a critical review of the study material
          \item Aggregative: the summary should put different source texts in relation to one another and by doing this give an overview of a certain topic
          \item Other (please specify)
        \end{itemize}

        \\

        Q38 & \textbf{Is there anything else you would want us to know about your ideal summary that we have not covered yet?} \\

        & Open response

        \\

        Q39 & \textbf{Is there anything else you would want us to know about the original study material that we have not covered yet?} \\

        & Open response

        \\ \midrule

        \multicolumn{2}{l}{\textbf{Look out questions}} \\ \midrule

        & Now, let's assume the pre-made summary was generated by a computer. You can assume that this machine generated summary captures all the needs you have identified in the previous questions. \\

        Q40 & \textbf{Would it make a difference to you whether the summary was generated by a computer program or by a human?}
        \begin{itemize}[label=$\square$, leftmargin=*, nosep]
          \item Yes \textit{-- participants are led to Q41}
          \item No \textit{-- participants are led to Q43}
        \end{itemize}

        \\

        Q41 & \textbf{Please explain the difference.} \\

        & Open response

        \\

        Q42 & \textbf{Which type of summary would you trust more:}
        \begin{itemize}[label=$\square$, leftmargin=*, nosep]
          \item A summary generated by a human, for example a teacher or a good performing fellow student
          \item A summary generated by a computer
          \item No difference
        \end{itemize}
        \\

        Q43 & \textbf{Please explain your answer.} \\

        & Open response

        \\

        Q44 & \textbf{Which type of summary would you trust more:}
        \begin{itemize}[label=$\square$, leftmargin=*, nosep]
          \item A summary generated by a human, for example a teacher or a good performing fellow student
          \item A summary generated by a computer
          \item No difference
        \end{itemize}

        \\

        & Now imagine that you can interact with the computer program that made the summary, in the form of a digital assistant. Imagine that your digital assistant made an initial summary for you and you can ask questions about it to your digital assistant and the assistant can answer them. Answers can be voice output, but also screen output, e.g. a written summary on the screen. In the next part we would like to investigate how you would interact with the assistant. Please do not feel restricted by the capabilities of today's digital assistants. \\

        Q45 & \textbf{Please choose the three most useful features for a digital assistant to have in this scenario.}
        \begin{itemize}[label=$\square$, leftmargin=*, nosep]
          \item Summarize particular parts of the study material with more detail
          \item Summarize particular parts of the study material with less detail
          \item Switch between different summary styles (for example highlighting vs a generated small piece of text)
          \item Explain why particular pieces ended up in the summary
          \item Provide the source of certain parts of the summary on request
          \item Search for different related sources based on the content of the summary
          \item Answer specific questions based on the content of the summary
        \end{itemize}

        \\

        Q46 & \textbf{Please choose the three least useful features for a digital assistant to have in this scenario.}
        \begin{itemize}[label=$\square$, leftmargin=*, nosep]
          \item Summarize particular parts of the study material with more detail
          \item Summarize particular parts of the study material with less detail
          \item Switch between different summary styles (for example highlighting vs a generated small piece of text)
          \item Explain why particular pieces ended up in the summary
          \item Provide the source of certain parts of the summary on request
          \item Search for different related sources based on the content of the summary
          \item Answer specific questions based on the content of the summary
        \end{itemize}

        \\

        Q47 & \textbf{Can you think of any other features that you would like your digital assistant to have to help you in this scenario?} \\

        & Open response

        \\ \midrule

        \multicolumn{2}{l}{\textbf{Background questions}} \\ \midrule

        & Thank you for filling out this survey so far! We would still like to ask you two final background questions. \\

        Q48 & \textbf{What is the current level of education you are pursuing?}
        \begin{itemize}[label=$\square$, leftmargin=*, nosep]
          \item Bachelor's degree
          \item Master's degree
          \item MBA
          \item Other, please specify
        \end{itemize}

        \\

        Q49 & \textbf{What is your main field of study?} \\

        & Open response \\

        \\ \midrule

        \multicolumn{2}{l}{\textbf{Thank you!}} \\ \midrule

        & You have come to the end of our survey. Thanks a lot for helping out! We very much appreciate your time. \\

        Q50 & \textbf{If you would like to participate in the raffle to win a voucher, please fill out your e-mail address below. We will only use this e-mail address to blindly draw 10 names who win a voucher and to contact you if your name has been drawn.} \\

        & Open response


\end{xtabular}

\twocolumn
